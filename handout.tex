%!TEX TS-program = xelatex
%!TEX TS-options = -synctex=1
%!TEX encoding = UTF-8 Unicode
%
%  handout realappearance 
%
%  Created by Mark Eli Kalderon on 2013-01-04.
%  Copyright (c) 2013. All rights reserved.
%

\documentclass[12pt]{article} 

% Definitions
\newcommand\mykeywords{Cook Wilson, perception, realism, appearance} 
\newcommand\myauthor{Mark Eli Kalderon} 
\newcommand\mytitle{Realism and Perceptual Appearance}

% Packages
\usepackage{geometry} \geometry{a4paper} 
\usepackage{url}
\usepackage{txfonts}
\usepackage{enumerate}
\usepackage{color}
\definecolor{gray}{rgb}{0.459,0.438,0.471}

% XeTeX
\usepackage{fontspec}
\usepackage{xltxtra,xunicode}
\defaultfontfeatures{Scale=MatchLowercase,Mapping=tex-text}
\setmainfont{Hoefler Text}
\setsansfont{Gill Sans}
\setmonofont{Inconsolata}

% Bibliography
\usepackage[round]{natbib} 

% Title Information
\title{\mytitle} % For thanks comment this line and uncomment the line below
% \title{\mytitle\thanks{}}% 
\author{\myauthor} 
\date{} % Leave blank for no date, comment out for most recent date

% PDF Stuff
% \usepackage[plainpages=false, pdfpagelabels, bookmarksnumbered, backref, pdftitle={\mytitle}, pagebackref, pdfauthor={\myauthor}, pdfkeywords={\mykeywords}, xetex, colorlinks=true, citecolor=gray, linkcolor=gray, urlcolor=gray]{hyperref} 

%%% BEGIN DOCUMENT
\begin{document}

% Title Page
\maketitle

% Main Content

% Layout Settings
\setlength{\parindent}{1em}

\begin{quote}
	About ten men were present at the first informal class of that year. He was treating by request the Kantian paradox: `the mind makes nature, the material it does not make'. He paused in his familiar manner and bending forward looked fixedly in the face of a Balliol man in a ragged scholar's gown. He, supposing himself to be interrogated or in a spirit of mischief, blurted out: `But why shouldn't the table be there, just where we see it?' Silence attended the result. The professor sprang once in the air; said very fiercely indeed: `Why shouldn't it?' and then relapsed into reverie. The scholar never returned, but I have sometimes wondered whether the shock set Wilson determinedly to work clearing the path which after many days led him far from the idealist solution he then accepted or appeared to accept. (Farquharson, \emph{Memoir}, in \citealt[]{Cook-Wilson:1926sf})
\end{quote}

Four key features of Cook Wilson's realist epistemology:
\begin{enumerate}% [(1)]
	\item The objects of knowledge are independent of the subject's knowledge of them
	\item In knowing something, the subject possesses a warrant akin to proof
	\item There is no theory of knowledge
	\item \emph{The accretion}: A subject's state of knowledge could not be indistinguishable upon reflection from a state in which one seems to have proof sufficient for knowledge but in fact lacks such proof
\end{enumerate}

\begin{quote}
	You can no more act upon the object by knowing it than you can `please the Dean and Chapter by stroking the dome of St. Paul's'. The man who first discovered the equable curvature meant equidistance from a point didn't supposed that he `produced' the truth---that absolutely contradicts the idea of truth---nor that he changed the nature of the circle or curvature, or of the straight line, or of anything spatial. \citep[802]{Cook-Wilson:1926sf}
\end{quote}

\begin{quote}
	In knowing, we can have nothing to do with the so-called `greater strength' of the evidence on which the opinion is grounded; simply because we know that this `greater strength' of evidence of A's being B is compatible with A's not being B after all. \ldots Belief is not knowledge and the man who knows does not believe at all what he knows; he knows it. \citep[100]{Cook-Wilson:1926sf}
\end{quote}

\begin{quote}
	I believe that, in practice, most people agree with John Locke that ``the certainty of things existing \emph{in rerum natura}, when we have the testimony of our sense for it, is not only as great as our frame can attain to, but as our condition needs.'' \citep[1]{Ayer:1958kx}
\end{quote}

\begin{quote}
	Perhaps most fallacies in the theory of knowledge are reduced to the primary one of trying to explain the nature of knowledge or apprehending. We cannot construct knowing---the act of apprehending---out of any elements. I remember quite early in my philosophic reflection having an instinctive aversion to the very expression `theory of knowledge'. I felt the words themselves suggested a fallacy---an utterly fallacious inquiry, though I was not anxious to proclaim <it>. \citep[803]{Cook-Wilson:1926sf}
\end{quote}

\begin{quotation}
	\noindent \emph{Note upon a certain confusion to which we are liable in regard to the conception of appearance}
	
	If we perceive some property of an object, there is presupposed on the one hand the property of the object as existing in its own account whether we perceive it or not; and as distinct from this, our act of perceiving or recognizing the nature of this property.
	
	This latter, the subjective act of ours, is sometimes spoken of from the side of the object as the \emph{appearance} of the object to us. This `appearance' then gets distinguished from the object, and that in itself is justified in so far as our subjective act of recognition of the object's nature is not the same kind as that nature. But next the \emph{appearance}, though properly the appear\emph{ing} of the object, gets to be looked upon as itself an object and the immediate object of consciousness, and being already, as we have seen, distinguished from the subject and related to our subjectivity, becomes, so to say, a merely subjective `object'---`appearance' in that sense. And s, as \emph{appearance} of the object, it has now to be represented not as the object but as some phenomenon caused in our consciousness by the object. Thus for the true appearance (= appearing) to us of the \emph{object} is substituted through the `objectification' of the appearing as appearance, the appearing to us of an \emph{appearance}, the appearing of a phenomenon caused in us by the object. (The thing to emphasize on the contrary is that the so-called appearance is the appearing of the \emph{object}, that is, we have the nature of the object before us and not only some affection of our consciousness produced by it.)

	It must be observed that the result of this is that there could be no direct perception or consciousness of Reality under any circumstances or any condition of knowing or perceiving; for the whole view is developed entirely from the fact that the object is distinct from our act of knowing it or recognizing it, which distinction must exist in any kind of knowing it or perceiving it. From this error would necessarily result a mere subjective idealism. Reality would become an absolutely unknowable `Thing in Itself', and finally disappear altogether (as with Berkeley) as an hypothesis which we couldn't possibly justified. \citep[796-797]{Cook-Wilson:1926sf}
\end{quotation}

\begin{quote}
    The cause of Sense, is the External Body, or Object, which presseth the organ proper to each Sense, either immediately, as in the Tast and Touch; or mediately, as in the mediation of Nerves, and other strings, and membranes of the body, continued inwards to the Brain, and Heart, causeth there a resistance, or counter-pressure, or endeavour of the heart, to deliver it self; which endeavour because \emph{Outward}, seemeth to be some matter without. And this \emph{seeming}, or \emph{fancy}, is that which men call \emph{Sense}. (Hobbes, \emph{Leviathan} \textsc{i}.1)
\end{quote}

\begin{quote}
    But their appearance to us is Fancy, the same waking, that dreaming. And as pressing, rubbing, or striking the Eye, makes us fancy a light; and pressing the Eare, produceth a dinne; so do the bodies we see, or hear, produce the same by their strong, though unobserved action. (Hobbes, \emph{Leviathan} \textsc{i}.1)
\end{quote}

\begin{quote}
    For if these Colours, and Sounds, were in the Bodies, or Objects that cause them, they could not bee severed from them, as by glasses, and in Ecchoes by reflection, wee see they are; where we know the thing we see, is one place; the apparence, in another. (Hobbes, \emph{Leviathan} \textsc{i}.1)
\end{quote}

\begin{quote}
	\ldots\ if perceiving were a kind of knowing, mistakes about what we perceive would be impossible, and yet they are constantly being made, since at any rate in the cases of seeing and feeling or touching we are almost always in a state of thinking that what we are perceiving are various bodies, although we need only to reflect to discover that in this we are mistaken. \citep[11]{Prichard:1938ve}
\end{quote}

\nocite{Hobbes:1651fk}

% Bibligography
\bibliographystyle{plainnat} 
\bibliography{Philosophy} 

\end{document}