%!TEX TS-program = xelatex
%!TEX TS-options = -synctex=1
%!TEX encoding = UTF-8 Unicode
%
%  realappearance
%
%  Created by Mark Eli Kalderon on 2012-12-22.
%  Copyright (c) 2012. All rights reserved.
%

\documentclass[12pt]{article} 

% Definitions
\newcommand\mykeywords{Cook Wilson, perception, realism, appearance} 
\newcommand\myauthor{Mark Eli Kalderon} 
\newcommand\mytitle{Realism and Perceptual Appearance}

% Packages
\usepackage{geometry} \geometry{a4paper} 
\usepackage{url}
\usepackage{txfonts}
\usepackage{color}
\definecolor{gray}{rgb}{0.459,0.438,0.471}

% XeTeX
\usepackage{fontspec}
\usepackage{xltxtra,xunicode}
\defaultfontfeatures{Scale=MatchLowercase,Mapping=tex-text}
\setmainfont{Hoefler Text}
\setsansfont{Gill Sans}
\setmonofont{Inconsolata}

% Bibliography
\usepackage[round]{natbib} 

% Title Information
\title{\mytitle} % For thanks comment this line and uncomment the line below
% \title{\mytitle\thanks{}}% 
\author{\myauthor} 
\date{} % Leave blank for no date, comment out for most recent date

% PDF Stuff
% \usepackage[plainpages=false, pdfpagelabels, bookmarksnumbered, backref, pdftitle={\mytitle}, pagebackref, pdfauthor={\myauthor}, pdfkeywords={\mykeywords}, xetex, colorlinks=true, citecolor=gray, linkcolor=gray, urlcolor=gray]{hyperref} 

%%% BEGIN DOCUMENT
\begin{document}

% Title Page
\maketitle
% \begin{abstract} % optional
% \noindent
% \end{abstract} 
\vskip 2em \hrule height 0.4pt \vskip 2em
% Main Content
% \epigraph{}

% Layout Settings
\setlength{\parindent}{1em}

\section{The Balliol Man in a Ragged Scholar's Gown} % (fold)
\label{sec:the_balliol_man_in_a_ragged_scholar_s_gown}

According to a familiar story, analytic philosophy begins with a revolt against idealism, the rebellion beginning principally in Cambridge and subsequently spreading throughout the anglophone world. Like nativity stories generally, it is idealized and suits the purposes of those who would propound it. One problem with the story is that it does injustice to the variety of sources from which analytic philosophy in its formative stage drew. Even when we grant, as we must surely do, that among the sources of analytic philosophy, a reaction against idealism is both important and salient, it can seem parochial if not provincial to see the revolt against idealism exclusively from the point of view of Cambridge. Importantly, roughly contemporaneously if not slightly before, doubts about idealism were circulating in Oxford. 

Farquharson in his \emph{Memoir} of John Cook Wilson, reports the following anecdote:
\begin{quote}
	About ten men were present at the first informal class of that year. He was treating by request the Kantian paradox: `the mind makes nature, the material it does not make'. He paused in his familiar manner and bending forward looked fixedly in the face of a Balliol man in a ragged scholar's gown. He, supposing himself to be interrogated or in a spirit of mischief, blurted out: `But why shouldn't the table be there, just where we see it?' Silence attended the result. The professor sprang once in the air; said very fiercely indeed: `Why shouldn't it?' and then relapsed into reverie. The scholar never returned, but I have sometimes wondered whether the shock set Wilson determinedly to work clearing the path which after many days led him far from the idealist solution he then accepted or appeared to accept.
\end{quote}
The anecdote is potentially apocryphal. That it takes the form of a conversion narrative can suggest this---though this is not the first conversion narrative in philosophy, consider the moral conversion that Kant undergoes upon reading Rousseau.

% section the_balliol_man_in_a_ragged_scholar_s_gown (end)

% Bibligography
% \bibliographystyle{plainnat} 
% \bibliography{Philosophy} 

\end{document}