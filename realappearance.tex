%!TEX TS-program = xelatex
%!TEX TS-options = -synctex=1
%!TEX encoding = UTF-8 Unicode
%
%  realappearance
%
%  Created by Mark Eli Kalderon on 2012-12-22.
%  Copyright (c) 2012. All rights reserved.
%

\documentclass[12pt]{article} 

% Definitions
\newcommand\mykeywords{Cook Wilson, perception, realism, appearance} 
\newcommand\myauthor{Mark Eli Kalderon} 
\newcommand\mytitle{Realism and Perceptual Appearance}

% Packages
\usepackage{geometry} \geometry{a4paper} 
\usepackage{url}
\usepackage{txfonts}
\usepackage{enumerate}
\usepackage{epigraph}
\usepackage{color}
\definecolor{gray}{rgb}{0.459,0.438,0.471}

% XeTeX
\usepackage{fontspec}
\usepackage{xltxtra,xunicode}
\defaultfontfeatures{Scale=MatchLowercase,Mapping=tex-text}
\setmainfont{Hoefler Text}
\setsansfont{Gill Sans}
\setmonofont{Inconsolata}

% Bibliography
\usepackage[round]{natbib} 

% Title Information
\title{\mytitle} % For thanks comment this line and uncomment the line below
% \title{\mytitle\thanks{}}% 
\author{\myauthor} 
\date{} % Leave blank for no date, comment out for most recent date

% PDF Stuff
\usepackage[plainpages=false, pdfpagelabels, bookmarksnumbered, backref, pdftitle={\mytitle}, pagebackref, pdfauthor={\myauthor}, pdfkeywords={\mykeywords}, xetex, colorlinks=true, citecolor=gray, linkcolor=gray, urlcolor=gray]{hyperref} 

%%% BEGIN DOCUMENT
\begin{document}

% Title Page
\maketitle
\begin{abstract} % optional
\noindent In his 1904 letter to G.F. Stout, Cook Wilson distinguishes objective and subjective conceptions of appearance and provides a diagnosis for the modern acceptance of the subjective conception in terms of a confused misdescription of the objective appearances that perceptual experience affords. Moreover, Cook Wilson links subjective appearances with idealism, the suggestion being that perceptual appearances must be objective if they are to afford us with something akin to proof of a world without the mind.
\end{abstract} 
% \vskip 2em \hrule height 0.4pt \vskip 2em
% Main Content
\epigraph{``A careful examination of the stars \ldots\ reveals to us the most startling appearances.''}{J. N. Lockyer, \emph{Elem. Lessons Astron.} i.iii.18 1879}

% Layout Settings
\setlength{\parindent}{1em}

\section{The Man from Balliol} % (fold)
\label{sec:the_balliol_man_in_a_ragged_scholar_s_gown}

According to a familiar story, analytic philosophy begins with a revolt against idealism, the rebellion beginning principally in Cambridge and subsequently spreading throughout the anglophone world. Like nativity stories generally, it is idealized and suits the purposes of those who would propound it. One problem with the story is that it does injustice to the variety of sources from which analytic philosophy, in its formative stages, drew. Even when we grant, as we must surely do, that among the sources of analytic philosophy, a reaction against idealism is both important and salient, it can seem parochial if not indeed provincial to see the revolt against idealism exclusively from the point of view of Cambridge. Importantly, roughly contemporaneously if not slightly before, doubts about idealism were circulating in Oxford. 

While Cambridge realism was more influential, at least early on, Oxford realism has proved more resilient, a ``tradition despite itself'', as \citet[xii]{Travis:1989to} wryly comments. Cambridge realists, such as \cite{Russell:1912uq}, \citet{Moore:1953nx}, and \citet{Price:1932fk}, maintained that perception affords us with a sensory mode of awareness and that we our knowledgeable of the mind-independent environment by being aware of that environment. They also maintained that \emph{all} sense experience, and not just perception, involves this sensory mode of awareness. Cambridge realists were thus committed to a kind of experiential monism (in Snowdon's \citeyear{Snowdon:2008oz} terminology): All sense experience involves, as part of its nature, a sensory mode of awareness. Even subject to illusion or hallucination, there is something of which one is aware. And with that, they were an application of the argument from illusion, or hallucination, or conflicting appearances away from immaterial sense data and a representative realism that tended, over time, to devolve into a form of phenomenalism. In contrast, the Oxford realists, Cook Wilson and early Prichard, restricted sensory awareness to perception, thus avoiding experiential monism and its degenerative effects. 

Most contemporary analytic philosophers believe neither in sense data, representative realism, nor phenomenalism. While realistically inclined, these aspects of Cambridge realism have largely been abandoned. Instead of sense data, contemporary philosophers of perception are more likely to speak of accuracy conditions or the representational content of perception \citep[though see][]{Robinson:1994ms}. And while perception is deemed to have a representational content, they will insist that this is insufficient for a representative realism. Rather, one is directly aware of the environment by one's perception accurately representing that environment. Moreover, few remain with phenomenalist sympathies (though see \citealt{Foster:2000ny} and \citealt{Noe:2004fk}). Nevertheless, \citet{Putnam:1994kx} has suggested that this contemporary orthodoxy has perhaps more in common with the sense-data theory, representative realism, and phenomenalism than they care to admit. It struck Putnam, at least at the time of the Dewey Lectures, as insufficiently demanding by the standards of James' natural realism that he admired and, like Cambridge realism, ultimately unsustainable. Putnam cites the work of \citet{McDowell:1994am} as exemplifying a more robust and sustainable form of realism. McDowell is one of the more recent figures in the tradition of Oxford realism that has quietly persisted as minority position in epistemology and philosophy of perception. Thinkers in this tradition despite itself include \citet{Cook-Wilson:1926sf}, \citet{Prichard:1950tg}, \citet{Ryle:1949qr}, \citet{Austin:1946ao,Austin:1962lr}, \citet{Hinton:1973js}, and more recently, \citet{McDowell:1994am}, \citet{Williamson:2000lr}, and \citet{Travis:2008la}. As we shall see, Putnam's remarks in the Dewey Lectures are anticipated by Cook Wilson in his letter to Stout.

Farquharson in his \emph{Memoir} of Cook Wilson, reports the following anecdote:
\begin{quote}
	About ten men were present at the first informal class of that year. He was treating by request the Kantian paradox: `the mind makes nature, the material it does not make'. He paused in his familiar manner and bending forward looked fixedly in the face of a Balliol man in a ragged scholar's gown. He, supposing himself to be interrogated or in a spirit of mischief, blurted out: `But why shouldn't the table be there, just where we see it?' Silence attended the result. The professor sprang once in the air; said very fiercely indeed: `Why shouldn't it?' and then relapsed into reverie. The scholar never returned, but I have sometimes wondered whether the shock set Wilson determinedly to work clearing the path which after many days led him far from the idealist solution he then accepted or appeared to accept. (Farquharson, \emph{Memoir}, in \citealt[xix]{Cook-Wilson:1926sf})
\end{quote}
The anecdote is potentially apocryphal. That it takes the form of a conversion narrative can suggest this---though this is not the first conversion narrative in philosophy, consider the moral conversion that Kant undergoes upon reading Rousseau. The anecdote, apocryphal or no, usefully highlights two themes whose interconnections we shall be exploring. The first is the fundamental realist commitment, common to Cook Wilson and \citet{Moore:1903uo}, that the objects of knowledge are independent of our knowledge of them. The second is that, at least in propitious circumstances, perception can make us knowledgeable of its objects. The Balliol man in the ragged scholar's gown challenges Cook Wilson by suggesting that we can know that the table is there, independently of us. That is the fundamental realist commitment. But he also suggests that we can come to have knowledge of a world that exists independently of us in a certain way, by perceiving it. We can know that the table is there, \emph{just where we see it}. The question that we shall be pursuing is: What must perception be like, according to Cook Wilson, if it is to make us knowledgeable of a mind-independent subject matter? 

Our answer will be partial and provisional. This is partly because I want to focus exclusively on one important piece of evidence, a crucial passage in Cook Wilson's letter to Stout. But importantly, it could turn out that any answer must, by the very nature of the case, be partial and provisional, since Cook Wilson may not have completely thought through the implications of his realist epistemology for the philosophy of perception. Indeed, it is arguable, and I have argued elsewhere \citep{Kalderon:2010fk}, that the implications for the philosophy of perception only fully gets worked out by subsequent thinkers in this tradition, notably by \citet{Prichard:1938ve}, \citet{Austin:1962lr}, and \citet{Hinton:1973js}.

We shall proceed as follows. 

First we shall review key features of Cook Wilson's realist epistemology. Perception makes us knowledgeable of its objects. To be knowledgable of the object of perception is for the subject to potentially know something about that object on the basis of their perceptual experience. Knowledgeableness is a matter of knowledge potential (in Williamson's \citeyear{Williamson:1990uq} terminology). It is knowledge potentially had. The knowledge potentially had by undergoing a perceptual experience must, by Cook Wilson's lights, exemplify the key features of his realist epistemology if it is to be knowledge at all. This imposes a substantive explanatory constraint on the nature of perception if it is to make us knowledgeable in this sense of its objects. What does perception have to be like in order to make us knowledgable of a mind-independent subject matter? 

Second, having laid out the background realist epistemology, we will look at a passage from a letter to Stout where Cook Wilson discusses the nature of perceptual appearances. In it, he contrasts objective and subjective conceptions of appearance, and provides a diagnosis for the modern acceptance of subjective appearance in terms of a confused misdescription of objective appearances. Moreover, and importantly for our purposes, he links, in this passage, subjective appearances and idealism, the suggestion being that perceptual appearances must be objective if they are to make us knowledgeable of a world without the mind.

% section the_balliol_man_in_a_ragged_scholar_s_gown (end)

\section{Knowledge and Perception} % (fold)
\label{sec:knowledge_and_perception}

Three features of Cook Wilson's realist epistemology will be relevant:
\begin{enumerate}[(1)]
	\item The objects of knowledge are independent of the subject's knowledge of them
	\item In knowing something, the subject possesses a warrant akin to proof
	\item There is no theory of knowledge
\end{enumerate}
Notoriously, there is a fourth feature, what Travis describes as \emph{the accretion}:
\begin{enumerate}[(4)]
	\item A subject's state of knowledge could not be indistinguishable upon reflection from a state in which one seems to have proof sufficient for knowledge but in fact lacks such proof
\end{enumerate}

Let's review these in turn.

\begin{enumerate}[(1)]
	\item First is the fundamental realist commitment common to Cook Wilson and \citet{Moore:1903uo}\----that the objects of knowledge are independent of the subject's knowledge of them. Suppose, as the Balliol man in the ragged scholar's gown contends, that seeing something in plain view can, in propitious circumstances, put you in a position to know various things about it. So you can know that the table is there just where you see it. The table is the object of your knowledge in the sense that you can know something about \emph{it}---about its age, shape, height, relative position, color, material composition, design, state of repair, and so on. According to the fundamental realist commitment, the table is the object of a subject's knowledge only insofar as it exists independently of the subject's knowledge of it. The object of knowledge must be independent of the subjects knowledge of it, if coming to know is to be a genuine discovery:
		\begin{quote}
			You can no more act upon the object by knowing it than you can `please the Dean and Chapter by stroking to dome of St. Paul's'. The man who first discovered that equable curvature meant equidistance from a point didn't supposed that he `produced' the truth---that absolutely contradicts the idea of truth---nor that he changed the nature of the circle or curvature, or of the straight line, or of anything spatial. \citep[\emph{Correspondence with Prichard 1904},][802]{Cook-Wilson:1926sf}
		\end{quote}
	\item Second is the conception of epistemic warrant as proof. For a subject to possess proof is for a subject to possess something inconsistent with the falsity of the conclusion. So if in order for the subject to know something, they must possess proof of what they know, they must possess something inconsistent with the falsity of what's known. If what warrant they possess falls short of proof but is at most very good evidence, then the subject could not know that thing, but could only believe it to the degree warranted by the evidence: 
		\begin{quote}
			In knowing, we can have nothing to do with the so-called `greater strength' of the evidence on which the opinion is grounded; simply because we know that this `greater strength' of evidence of A's being B is compatible with A's not being B after all. \ldots Belief is not knowledge and the man who knows does not believe at all what he knows; he knows it. \citep[\emph{Statement and Inference} \textsc{ii}.3, \emph{Opinion, Conviction, Belief and Cognate States},][100]{Cook-Wilson:1926sf}
		\end{quote}
The Cook Wilsonian conception of epistemic warrant as proof thus stands opposed to the Lockean conception of knowledge that Ayer commends:
		\begin{quote}
			I believe that, in practice, most people agree with John Locke that ``the certainty of things existing \emph{in rerum natura}, when we have the testimony of our sense for it, is not only as great as our frame can attain to, but as our condition needs.'' \citep[\emph{Foundations of Empirical Knowledge},][1]{Ayer:1958kx}
		\end{quote}
The opposition to the Lockean conception of knowledge is pursued and elaborated in their own ways by \citet{Prichard:1950tg} and \citet{Austin:1946ao,Austin:1962lr} \citep[see][for discussion]{Kalderon:2010fk}. However, the Cook Wilsonian opposition grounded in a conception of epistemic warrant as proof itself has historical precedent. Specifically, we have here a replay of a key theme of an early modern dispute between Hobbes and Boyle on the epistemic status of experimental philosophy \citep[see][for discussion]{Shapin:1985ad}. Like Hobbes, Cook Wilson regards Euclidean geometry as the paradigm of knowledge. Cook Wilson, however, departs from Hobbes with respect to what is capable of proof. Specifically, Cook Wilson follows the Balliol man in the ragged scholar's gown in maintaining that we can have perceptual knowledge. If we can have perceptual knowledge, then what's perceived must be tantamount to proof of what's known on its basis. Cook Wilson maintains, and Hobbes denies, that the senses can, sometimes at least, provide us with such proof.
	\item Third is the Cook Wilsonian denial of a theory of knowledge. This is sometimes expressed by saying that it is a mistake to explain the nature of knowledge or apprehension. This can encourage the thought that Cook Wilson is endorsing a kind of quietism about the nature of knowledge. This, however, would be a misunderstanding:
		\begin{quote}
			Perhaps most fallacies in the theory of knowledge are reduced to the primary one of trying to explain the nature of knowledge or apprehending. We cannot construct knowing---the act of apprehending---out of any elements. I remember quite early in my philosophic reflection having an instinctive aversion to the very expression `theory of knowledge'. I felt the words themselves suggested a fallacy---an utterly fallacious inquiry, though I was not anxious to proclaim <it>. \citep[\emph{Correspondence with Prichard 1904},][803]{Cook-Wilson:1926sf}
		\end{quote}
This is a clear statement of the anti-hybridism or anti-conjunctivism about knowledge that \citet{McDowell:1982kx} and \citet{Williamson:2000lr} will later defend. So conceived, knowledge is not a hybrid state consisting of an internal, mental state and the satisfaction of some external conditions. Cook Wilson's aversion to the ``theory of knowledge'' is just an aversion to explaining knowledge by constructing it out of elements, and this skepticism will be echoed by \citet{Prichard:1950tg}, \citet{Ryle:1949qr}, and \citet{Austin:1962lr} and in precisely these terms.
	\item The fourth and final feature is the accretion. The accretion was undoubtedly thought to be a consequence of Cook Wilson's conception of epistemic warrant as proof. Specifically, the thought behind the accretion is that to be a proof is to, among other things, admit of no ringers. A merely apparent proof would be consistent with the falsity of its conclusion (otherwise it wouldn't be \emph{merely} apparent). If an apparent proof were indistinguishable upon reflection from a merely apparent proof, then reflection upon the apparent proof is potentially consistent with the falsity of its conclusion---which means that it is no proof at all. Within the Oxford tradition, \citet{Prichard:1938ve} will retain the accretion and reject perceptual realism, whereas \citet{Austin:1962lr} will retain perceptual realism and reject the accretion. The insight moving each was that the accretion is inconsistent with perception affording knowledge if this involves the possession of something akin to proof.
\end{enumerate}

These features of Cook Wilson's realist epistemology have implications for the nature of perception if we assume, as the Balliol man in the ragged scholar's gown insists we must, that perceptual experience makes us knowledgeable of its objects. To say that perception makes us knowledgeable about its object is to say that the perceiver can potentially know something about the perceived object on the basis of their experience of it. This potentiality need not be actualized. Nevertheless, there must be in perception the basis for potentially knowing something, if perception genuinely makes us knowledgeable of its objects.

With this assumption in place, consider, then, the first feature of Cook Wilson's realist epistemology---that the objects of knowledge are independent of the subject's knowledge of them. If we can only know about that which exists independently of our knowledge and apprehension, then if perception makes us knowledgeable about its objects, the objects of perception must themselves exist independently of being perceived, recognized, or known. If the objects of perception weren't existentially independent of the subject's perceptual experience, that experience wouldn't make the subject knowledgeable of the perceived object. This is a relatively straight forward example of how Cook Wilson's realist epistemology has implications for the philosophy of perception given the claim about the epistemic significance of perception---that perception makes the subject knowledgeable about its objects.

Similar remarks hold of Cook Wilson's conception of epistemic warrant as proof. If perception makes the subject genuinely knowledgeable about its object, then in perceiving that object the perceiver must possess a warrant akin to proof of what can be known on that basis. As we shall see, this imposes a substantive constraint on the nature of perception. Perception must be of such a nature that it is impossible to undergo that perceptual experience and any proposition potentially known on that basis be false. What must the metaphysics of perception be like in order for this to be so?

However, the main question that we shall be pursuing concerns the third element of Cook Wilson's realist epistemology---his denial that knowledge is constructed out of elements. Suppose we accept anti-hybridism or anti-conjunctivism about knowledge. Suppose that knowledge is an explanatorily fundamental and irreducible relation to its object. What must perception be like if it is to make us knowledgeable in this sense of its object? That will be the fugitive question that we shall be pursuing in carefully reviewing the crucial passage from Cook Wilson's letter to Stout.


% section knowledge_and_perception (end)

\section{Appearance} % (fold)
\label{sec:appearance}

Cook Wilson's letter to Stout was written in response to Stout's \citeyearpar{Stout:1903zl} presentation to the Aristotelian Society, ``Primary and secondary qualities.'' Cook Wilson's correspondence was part of a larger controversy that Stout's presentation gave rise to \citep[see][for how this debate played out in Edwardian philosophy]{Nassim:2008fk}. The letter is the most detailed extant discussion we have from Cook Wilson about the nature of perception. It is long. Easily longer than Stout's original paper. And it is packed with argument and rich with ideas. Today I want to just begin to explore Cook Wilson's views about the nature of perception by taking a close look at the following brief passage:
\begin{quotation}
	\noindent \emph{Note upon a certain confusion to which we are liable in regard to the conception of appearance}
	
	If we perceive some property of an object, there is presupposed on the one hand the property of the object as existing in its own account whether we perceive it or not; and as distinct from this, our act of perceiving or recognizing the nature of this property.
	
	This latter, the subjective act of ours, is sometimes spoken of from the side of the object as the \emph{appearance} of the object to us. This `appearance' then gets distinguished from the object, and that in itself is justified in so far as our subjective act of recognition of the object's nature is not the same kind as that nature. But next the \emph{appearance}, though properly the appear\emph{ing} of the object, gets to be looked upon as itself an object and the immediate object of consciousness, and being already, as we have seen, distinguished from the subject and related to our subjectivity, becomes, so to say, a merely subjective `object'---`appearance' in that sense. And so, as \emph{appearance} of the object, it has now to be represented not as the object but as some phenomenon caused in our consciousness by the object. Thus for the true appearance (= appearing) to us of the \emph{object} is substituted through the `objectification' of the appearing as appearance, the appearing to us of an \emph{appearance}, the appearing of a phenomenon caused in us by the object. (The thing to emphasize on the contrary is that the so-called appearance is the appearing of the \emph{object}, that is, we have the nature of the object before us and not only some affection of our consciousness produced by it.)

	It must be observed that the result of this is that there could be no direct perception or consciousness of Reality under any circumstances or any condition of knowing or perceiving; for the whole view is developed entirely from the fact that the object is distinct from our act of knowing it or recognizing it, which distinction must exist in any kind of knowing it or perceiving it. From this error would necessarily result a mere subjective idealism. Reality would become an absolutely unknowable `Thing in Itself', and finally disappear altogether (as with Berkeley) as an hypothesis which we couldn't possibly justified. \citep[\emph{Correspondence with Stout 1904},][796-797]{Cook-Wilson:1926sf}
\end{quotation}

The passage consists of four paragraphs. The first paragraph distinguishes the object of perception and the episode of perceiving it, the second paragraph provides a diagnosis for the modern acceptance of subjective appearances in terms of the advertised confusion that we are liable in regard to the conception of appearance, the third paragraph links objective appearances with realism or at least anti-idealism and the fourth paragraph summarizes the conclusion of this section.

If a subject perceives some property of an object---say, the redness of the tomato ---then the property that the subject sees, the redness of the tomato, exists on its own account whether the subject perceives it or not. So the perceived property instance, the redness of the tomato, is existentially independent of the subject's experience of it. Moreover, potentially at least, the perceived property instance exists prior to the subject's perceptual experience. Given its existential independence and potential priority, the object of perception, the perceived redness of the tomato, is necessarily distinct from any act of perceiving or recognizing the nature of this property. This is the main conclusion of the first paragraph---that the object of perception exists independently and potentially prior to the subject's perceptual experience which is necessarily distinct from this object. As will emerge in the third paragraph, the distinction between the object of perception and the act of perception, understood as the subject's perceptual experience of the object, is a consequence of the very nature of perception, understood as a species of apprehension: One can only apprehend, whether in perception or knowledge, what exists independently of that apprehension.

Having distinguished the object of perception---that which is apprehended in perception---and the act of per\-cep\-tion\----the apprehending of the object in per\-cep\-tion\----Cook Wilson distinguishes objective and subjective conceptions of appearance. The subjective conception of appearance is familiar from early modern philosophy. Cook Wilson mentions Berkeley, but plausibly Hobbes is and advocate as well. In the second paragraph, Cook Wilson provides a diagnosis for subjective appearances in terms of a confusion we are liable in regard to the conception of appearance. 

That the object of perception is necessarily distinct from a subject's perceptual experience of it is consistent with that object being a constituent of the subject's perceptual experience, nonetheless. Perception, according to Cook Wilson, is a species of apprehension. In perceiving an object, the subject apprehends that object in their perceptual experience of it. But apprehension is relational. A perceptual experience could only be an apprehension, if there were something apprehended. So the object of perception, the object apprehended in the subject's perceptual experience, is plausibly a constituent of the subject's perceptual experience understood as the apprehension in perception of that object. Since the act of perception, the subject's perceptual experience of the object, is understood in this way, as the apprehension of the object in perception and so relational, it is natural speak of the object's appearance to the subject. When the subject perceives the redness of the tomato, the redness of the tomato appears to the subject. The appearance of the object of perception just is its presentation in the episode of perceptual apprehension. This is the objective conception of appearance. An object objectively appears when it is apprehended in the subject's perceptual experience of it. The conception of appearance is objective since what appears, the object of perception, exists independently and potentially prior to the subject's perceptual apprehension of it. It is the appearing of an independently existing object in the subject's perceptual experience.

Beginning with objective perceptual appearances, Cook Wilson explains how one might arrive at a subjective conception of appearance, familiar from early modern philosophy, understood as a phenomenon caused in our consciousness by the object, an affectation of the subject's consciousness produced by the object of perception. Hobbes' account of the senses in chapter one of \emph{Leviathan} provides a clear example:
\begin{quote}
    The cause of Sense, is the External Body, or Object, which presseth the organ proper to each Sense, either immediately, as in the Tast and Touch; or mediately, as in the mediation of Nerves, and other strings, and membranes of the body, continued inwards to the Brain, and Heart, causeth there a resistance, or counter-pressure, or endeavour of the heart, to deliver it self; which endeavour because \emph{Outward}, seemeth to be some matter without. And this \emph{seeming}, or \emph{fancy}, is that which men call \emph{Sense}. \citep[\emph{Leviathan},][\textsc{i}.1]{Hobbes:1651fk}
\end{quote}
The conception of appearance, this seeming or fancy, is subjective since an appearance, so understood, is a modification of the conscious subject. The object of perception causes the subject to go into a conscious state. The relation between the object of perception and the affectation of the subject's consciousness is merely causal. If, as Hume maintained, cause and effect are contingently related, then the very same effect, the very same conscious episode elicited by the perceived tomato might have been produced by a very different cause. Thus Hobbes writes:
\begin{quote}
    But their appearance to us is Fancy, the same waking, that dreaming. And as pressing, rubbing, or striking the Eye, makes us fancy a light; and pressing the Eare, produceth a dinne; so do the bodies we see, or hear, produce the same by their strong, though unobserved action. \citep[\emph{Leviathan},][\textsc{i}.1]{Hobbes:1651fk}
\end{quote}

How does one arrive at the subjective conception of appearance from the objective appearances that our perceptual experiences afford? By means of a confusion we are liable in regard to the conception of appearance. Ironically, the confusion is the converse of a Humean projective error. A projective error involves mistaking an aspect of subjective appearances for an aspect of the world independently of its appearance in our perceptual experience of it. The conviction that redness inheres in the tomato independently of our perceiving it might be deemed a projective error if the redness were thought, instead, to be a quality of subjective appearances \citep[for a contemporary version of this view see][]{Boghossian-Velleman:1989af,Boghossian-Velleman:1991as}. The metaphor may be Hume's, but the attribution of projective error has earlier precedent. Thus Hobbes once again:
\begin{quote}
    For if these Colours, and Sounds, were in the Bodies, or Objects that cause them, they could not bee severed from them, as by glasses, and in Ecchoes by reflection, wee see they are; where we know the thing we see, is one place; the apparence, in another. \citep[\emph{Leviathan},][\textsc{i}.1]{Hobbes:1651fk}
\end{quote}
The confusion we are liable in regard to the conception of appearance is the converse of Humean projective error. To emphasize this, we might call it, though Cook Wilson does not, an introjective error. In it, we mistake an aspect of what objectively appears to us for an aspect of subjective appearances, the latter a misconception arising from just this confusion.

How does this introjective error arise? The object of perception exists independently and potentially prior to its appearance in the subject's perceptual experience. The object of perception, the redness of the tomato, is thus distinct from its appearance, understood as its appearing to the subject in their perceptual apprehension of it. This appearance, properly understood as the appearing of an independently existing object in the subject's perceptual experience, being thus distinguished from the object, is now regarded as itself an immediate object of consciousness. This appearance has already been distinguished from the independently existing object and related to our subjectivity, since it is an appearing to a subject of the object. If it is now regarded as an immediate object of consciousness, the result is the subjective conception of appearances, the appearing of a conscious phenomenon in the subject as produced by the object:
\begin{quote}
    And so, as \emph{appearance} of the object, it has now to be represented not as the object but as some phenomenon caused in our consciousness by the object. Thus for the true appearance (=appearing) to us of the \emph{object} is substituted through the `objectification' of the appearing as appearance, the appearing to us of an \emph{appearance}, the appearing of a phenomenon cause in us by the object. \citep[\emph{Correspondence with Stout 1904},][796]{Cook-Wilson:1926sf}
\end{quote}
So the introjective error proceeds by the objectification of the appearing as appearance. Once the appearing of the object, properly distinguished from the independently existing object and related to our subjectivity, is regarded as the immediate object of consciousness, we are liable to mistake aspects of what objectively appears in our perceptual experience for aspects of subjective experiences. Thus in regarding redness not as a quality that inheres in bodies, such as ripened tomatoes, independently of our perception of them, but as a quality of our experience of ripened tomatoes, theorists such as Boghossian and Velleman have, by Cook Wilson's lights, themselves committed an introjective error. They mistake a quality of the object of our perceptual experience for a quality of a conscious experience produced in us by the perceived tomato.

Cook Wilson's idea that the subjective conception of appearance is the result of an introjective error has potential application to, and consequences for, the mind--body problem, understood as the hard problem of consciousness. The hard problem of consciousness asks how experience with a certain qualitative character could be materially realized? But suppose the quality of the conscious experience in terms of which the hard problem of consciousness is posed is really a quality of the perceived object. Then instead of asking Mary what the experience of red is like upon her first seeing a red thing, a ripe tomato, as it happens \citep{Jackson:1982my}, we should ask instead what red is like. What Mary learns from her first visual encounter with a red thing is not, or at least not primarily, what it is like to experience red things. Rather, Mary learns, in the first instance, what red things are like, what they are like in visible respects (compare Jackson's \citeyear{Jackson:1977fk} achromatic color picker). From this perspective, the hard problem of consciousness can seem like the problem of the manifest seen through the distorting effects of introjective error. (For further doubts about the mind--body problem along this line see \citealt{Shoemaker:2003wk}, \citealt{Byrne:2005jw}, and \citealt{Kalderon:2006tg}.)

There is nothing like an argument here for objective appearances. Rather, Cook Wilson, at this stage in his letter to Stout, presupposes that perception is a species of apprehension, and provides a diagnosis for the subjective conception of appearance in terms of a mistaken objectification of the objective appearing of the object in perceptual experience. The diagnosis is thus only as credible as the presupposed conception of perception and perceptual appearance.

However, a stronger theme emerges in the third paragraph of the passage, and one that anticipates some of Putnam's \citeyearpar{Putnam:1994kx} claims in the Dewey Lectures. There Cook Wilson suggests that unless perception were conceived in this way---as the apprehension of an object that exists independently of a subject's perceptual experience of it---then there could be no knowledge formed on the basis of perception:
\begin{quote}
    It must be observed that the result of this is that there could be no direct perception or consciousness of Reality under any circumstances or any condition of knowing or perceiving: for the whole view is developed entirely from the fact that the object is distinct from our act of knowing it or recognizing it, which distinction must exist in any kind of knowing it or perceiving it. From this error would necessarily result a mere subjective idealism. Reality would become an absolutely unknowable `Thing in Itself', and finally disappear altogether (as with Berkeley) as an hypothesis which we couldn't possibly justify. \citep[\emph{Correspondence with Stout 1904},][797]{Cook-Wilson:1926sf}
\end{quote}

Cook Wilson begins by avowing the fundamental realist commitment he shares with \citet{Moore:1903uo}. The object of apprehension, if it is to be apprehended at all, must exist independently of being apprehended. Since perception and knowledge are, according to Cook Wilson, both species of apprehension, it follows that the object of perception and the object of knowledge must each exist independently of being perceived, recognized, or known. Moreover we have seen how this claim about perception was entailed by the conjunction of the claim about knowledge---that the object of knowledge must exist independently of the subject's knowledge of that object---and a claim about the epistemic significance of perception---that perception makes the subject knowledgable about its object. If perception makes the subject knowledgeable about the perceived object in the sense that the subject potentially knows something about that object by perceiving it, then since objects of knowledge must be independent of the subject's knowledge of them, so too must the objects of perception. The Balliol man in the ragged scholar's gown insists that perception makes us knowledgable of a world without the mind. Cook Wilson's present point is that perception must be conceived in certain way, as the apprehension of the natural environment, if perceptions is to make us knowledgeable of that environment.

The first feature of Cook Wilson's realist epistemology is most prominently at play in the third paragraph---that the objects of knowledge are independent of the subject's knowledge of them. But arguably at least, the second and third features of Cook Wilson's epistemology are at play as well. 

So consider Cook Wilson's conception of epistemic warrant as proof. Cook Wilson follows Hobbes in this, and follows Hobbes, as well, in regarding this conception of epistemic warrant as paradigmatically exemplified by Euclidean geometry. Cook Wilson importantly departs from Hobbes in regarding perception as a source of epistemic warrant. The reason for this departure should be obvious from our previous discussion. It is Hobbes' conception of sensory experience, as merely a conscious modification of the perceiving subject, that is the obstacle to regarding perception as a source of a warrant akin to proof. If the conscious phenomenon caused in us by the object can arise from a variety of different causes, then undergoing a conscious affectation would be consistent with the nonexistence of the object that would have otherwise produced it. The very same visual impression of a red ripened tomato could have been produced not by the tomato that actually produced it but by the stimulation of the visual cortex by UCL neuroscientists, say. But if perception is to constitute a warrant akin to proof, perception could not be conceived in this way. If perceptual experience is to afford us proof about its objects, then that experience could not be the way that it in fact is if the propositions provably known on the basis of that experience were false. We see the tomato and are thus knowledgeable about that tomato. We could come to know that the tomato is red just by looking. What we see when we look must be akin to proof of the tomato's color. But if perception is a conscious modification of the perceiving subject, then our perceptual experience of the tomato could be what it is and the proposition that the tomato is red be false. On the subjective conception, perceptual appearances fall short of affording us proof of a world without the mind. As \citet{Locke:1706hc} observes, subjective appearances could afford us at most with a ``certainty \ldots\ only as great as our frame can attain to.''

% Our perception of the tomato wouldn't be what it is, the apprehension of the red of the tomato, if the tomato were some other color, orange or green, say. So, if perception is a mode of apprehension, our perceptual experience of the red tomato couldn't be what it is and the proposition that the tomato is red be false. 

What about the third feature of Cook Wilson's realist epistemology? Knowledge is not constructed out of elements. In this sense is there no theory of knowledge. Whatever else it may entail, a theory of knowledge would at least include conjunctive analyses of knowledge, that is, analyses that take the form of a conjunction of conditions on belief that constitutes knowledge, the kind of account familiar from the \emph{Theaetetus} and post-Gettier epistemology. So Cook Wilson is an anti-hybridist or anti-conjunctivist about knowledge in that knowledge does not consist in a justified belief that meets further external conditions. Knowledge, according to Cook Wilson, is an explanatorily fundamental and irreducible relation to its object. Moreover, this seems to be a more fundamental claim about the genus, apprehension. Apprehension, generally, is an explanatorily fundamental and irreducible relation to its object which exists independently of that apprehension. If that's right, and perception is a species of apprehension, then it would follow straight away that perception is explanatorily fundamental and irreducible relation to its object. Moreover, there is some reason to credit Cook Wilson with this line of thought. Thus Farquharsan claims that Cook Wilson came to regard the theory of perception absurd. If by theory of perception, Cook Wilson means something parallel to the theory of knowledge---such as the view, familiar from \citet{Grice:1962jw} and advocated by \citet{Hobbes:1651fk}, that perception is experience or subjective appearance that meets further external conditions such as being caused in the right sort of way by its veridical subject matter----then that would support the above reasoning. 

But is there another route to anti-conjunctivism about perception? One that flows from the demand that perception makes us knowledgable about its objects? Subjective appearances, if there were any, would be the experiential core common to perception, illusion, and hallucination. Subjective appearances are perceptual when they meet certain external conditions, when they are caused by, or counterfactually depend upon, in the right sort of way, their veridical subject matter, say. But the very same subjective appearance could have manifest itself in consciousness as the product of a very different cause. So subjective appearances, of the kind that Cook Wilson is keen to diagnose, would be the experiential core common to all sense experience. So understood, sense experience would have a single common nature, the subjective appearance that constitutes its experiential core. It is thus a kind of experiential monism. However, subjective appearances could only afford as much certainty as our frame can attain. Contrast the conception of perception that Cook Wilson advocates. On it, perception is, by its very nature, the apprehension in experience of its object. If the subject undergoes some other sensory experience which is not itself the apprehension of the object (either because it is the apprehension of some other object, or because it is the apprehension of no object at all), then it is not a perception of that object. Since perception essentially involves the apprehension of its object, then any sensory experience that is not an apprehension is not thereby a mode of perception. Far from being a kind of experiential monism, this can seem, instead, to be a kind of experiential pluralism. Perception is one kind of sense experience, the sense experience that the subject undergoes when apprehending in sensation aspects of the natural environment. Non-perceptual experiences, such as hallucinations, in apprehending nothing, must differ in nature from episodes such as perception, which are essentially apprehensions. Sense experience, so conceived, would lack a common experiential nature. Could perceptual apprehension, understood as an explanatorily fundamental and irreducible relation, afford us, not as much certainty as our frame can attain, but proof about its object?

Consider a speculative reason for embracing anti-conjunctivism about perception, one that flows from the demand that perception makes us knowledgable about its objects. Perhaps, if the object of perception is a \emph{constituent} of the subject's perceptual experience, then some headway, at least, is made in coming up with a perceptual warrant akin to proof. If the potential object of knowledge, being as it actually is, is a constituent of the perceptual experience, since it is the object apprehended in a perceptual act of apprehension, then the perceptual experience wouldn't be as it actually is, the apprehension of that object being as it is, if some observable truth about that object were false. My perception of the tomato wouldn't be the perceptual experience that it is, if the tomato apprehended were green, instead of red, say. It is because the potential object of knowledge is a constituent of the perceptual experience, by being perceptually apprehended, that the subject has a warrant akin to proof about the object of perceptual apprehension. And perhaps this can be coupled with the further thought that the object apprehended could only be a constituent of the subject's apprehension of it, in a manner that would have this epistemic significance, if apprehension were an explanatorily fundamental and irreducible relation. As applied to perception, understood as a sensory mode of apprehension, this would result in a kind of experiential pluralism. A hallucination would be an experience, but it would differ essentially from a perceptual experience, even one indiscriminable upon reflection from it, in being the apprehension of nothing. (For elaboration and defense of the first thought see \citealt{Kalderon:2011fk}; for elaboration and defense of the second thought see \citealt{Kalderon:2012fk}.)

One obstacle to Cook Wilson's thinking clearly through this may have been his adherence to the accretion. Consider the following argument of Prichard's \citeyearpar[11]{Prichard:1938ve}. Suppose a ripe tomato is in plain view of a subject, and that the subject can recognize as a tomato the fruit that they see. It might seem that what the subject thus apprehends is incompatible with there not being a tomato before them. In which case, perception affords the subject something akin to proof of the tomato's presence. In this way, perception can seem to make the subject knowledgeable of a mind-independent subject matter. Prichard's insight is that this picture is incompatible with a further feature of Cook Wilson's conception of knowledge, \emph{the accretion}. According to the accretion, if the subject knows that \emph{p}, the subject can know upon reflection that they know that \emph{p}. And if the subject has some attitude other than knowledge to that proposition, then the subject can know upon reflection that their attitude is something other than knowledge. Knowledge admits of no ringers---a state indiscriminable upon reflection from knowledge just is knowledge. What would it take for perception to make us knowledgeable of a mind-independent subject matter if there are no ringers for knowledge? If the subject's seeing the tomato makes them knowledgeable of the tomato's presence, then the subject must recognize that what they apprehends in seeing the tomato is incompatible with the tomato's absence. But is the subject in seeing the tomato in a position to recognize that? After all, there are situations indiscriminable upon reflection from seeing a tomato that does not involve the tomato's presence. The subject's hallucination of the scene would be indiscriminable upon reflection from their perceiving it. If what the subject apprehends in seeing the tomato is not discriminable upon reflection from what, if anything, they apprehend in hallucinating the tomato, then it could seem that they are not in a position to recognize that what they apprehend in seeing the tomato is incompatible with the tomato's absence. They would lack proof of a tomato before them. Thus Prichard concludes that since perception admits of ringers, it could not be a source of ringerless knowledge.

This argument reveals a tension within the Oxford realism inaugurated by Cook Wilson. If Cook Wilson was right in claiming that the objects of knowledge are mind-independent, and the objects of perception are at least potential objects of knowledge, then these claims can only be sustained by abandoning the accretion. Indeed, it is telling that \citet{Austin:1962lr} jettison's just this feature of Cook Wilson's epistemology. 

Against Ayer's \citeyearpar{Ayer:1958kx} claim that there is a type of sentence, an observation sentence, that represents how things appear in a subject's experience, that can be incorrigibly known to be true by the subject independently of the occasion of his expressing this knowledge, \citet{Austin:1962lr} insists that the truth of a claim is only determined by the standards in play on the occasion of utterance. If as Austin maintains, a sentence is only true when uttered on an occasion, there could be no sentence that is true independent of an occasion of utterance, and, hence, no such sentence could be incorrigibly known to be true. While no sentence can be incorrigibly known to be true independent of an occasion of utterance, that's not to say that there are no occasions where a subject can speak with certainty. But recognizing that there are occasions where things can be incorrigibly known undermines the thought that what can be incorrigibly known is restricted to reports about how things appear in sense experience:
\begin{quote}
    \ldots\ it may be said, even if such cautious formulae are not \emph{intrinsically} incorrigible, surely there will be plenty of cases in which what we say by their utterance will \emph{in fact} be incorrigible \ldots\ Well, yes, no doubt this is true. But then exactly the same thing is true of utterances in which quite different forms of word are employed \ldots\ if I watch for some time an animal a few feet in front of me, in a good light, if I prod it perhaps, and sniff, and take note of the noises it makes, I may say, `That’s a pig’; and this too will be `incorrigible’, nothing could be produced that would show that I had made a mistake. \citep[114--5]{Austin:1962lr}
\end{quote}
If circumstances are propitious, a subject can just know that there is a pig before them by seeing the pig. Seeing the pig and recognizing as a pig the animal that they see is incompatible with the pig's absence and so tantamount to proof of the pig's presence. So a subject can know there's a pig and can express this knowledge by saying ``There's a pig''. This is not undermined by there being other circumstances or other occasions where the very same sentence could be used to say something false and so fail to express knowledge. That there are other possible circumstances where a subject would speak falsely and fail to express knowledge (because they were hallucinating, say) is consistent with the subject, in the present circumstances, speaking truly and expressing knowledge of a pig before them. 

According to the accretion, that there are possible circumstances, where one lacks proof of what's ostensibly known, indiscriminable upon reflection from one's present circumstances, means that one lacks proof, and hence knowledge, in present circumstances, as well. Against this, Austin is insisting that one can have proof of a pig before one simply by seeing it, and that this is not undermined by there being other possible circumstances---such as seeing a hyperrealistic animatronic pig, or by having one's visual cortex appropriately stimulated by UCL neuroscientists, say---where one lacks such proof, even if these circumstances would be indiscriminable upon reflection from one's present circumstances.

\citet{Prichard:1938ve} retains the accretion and abandons perceptual realism, whereas \citet{Austin:1962lr} retains perceptual realism and abandons the accretion \citep[see][for discussion]{Kalderon:2010fk}. The insight moving each was that the accretion is inconsistent with perception affording knowledge of a world without the mind if this involves the possession of something akin to proof. So it is possible that the anti-conjunctivist implications of Cook Wilson's conception of perception are only fully made explicit in subsequent developments of this tradition despite itself, particularly in the work of \citet{Prichard:1938ve}, \citet{Austin:1962lr}, and \cite{Hinton:1973js}.

% section appearance (end)

\nocite{Hobbes:1651fk}

% Bibligography
\bibliographystyle{plainnat} 
\bibliography{Philosophy} 

\end{document}